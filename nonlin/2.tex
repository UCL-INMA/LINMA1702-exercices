\section{Conditions d'optimalité pour problèmes contraints}
\label{sec:nonlinApe2}  

\begin{enumerate}

  \item On cherche à déterminer parmi les points de la parabole
  d'équation $y=\frac{1}{5}(x-3)^2$ celui qui est le plus proche
  du point $(x,y)=(3,2)$, ce qu'on formule comme le
  problème d'optimisation suivant
  \[ \text{min} f(x,y) = (x-3)^2 + (y-2)^2 \text{ tel que } (x-3)^2 = 5y .\]
  \begin{enumerate}
    \item Pourquoi a-t-on utilisé comme fonciton objectif le \emph{carré}
    de la distance au point $(1,2)$ ?
    \item Trouvez tous les points satisfaisant à la condition (nécessaire)
    d'optimalité au premier ordre (en prenant soin de vérifier la condition 
    d'indépendance des gradients), et déduisez-en la solution du problème.
    \item Répondez à nouveau en utilisant cette fois la parabole 
    d'équation $y=\frac{1}{3}(x-3)^2$.
    \item Pouvait-on résoudre ce problème en éliminant $x$ à l'aide de la contrainte
    puis en résolvant le problème non-contraint ainsi obtenu ? 
    En éliminant $y$ ?
  \end{enumerate}
  
  \begin{solution}
    \begin{enumerate}
      \item Les opérations sont moins lourdes sur une somme de carrés 
      que sur une racine. Note qu'on peut le faire seulement parce que
      la fonction racine carrée est une fonction monotone.
      \item On a un système à trois équations 
      \begin{align*}
               2(x-3) &= 2\lambda (x-3) \\ 
               2(y-2) &= -5\lambda \\
               (x-3)^2 - 5y &= 0
      \end{align*}
      On a soit $x\neq 3$ et il n'y pas de solution,
      soit $x=3$ et $y=2$.
      \item On a les solutions $(3,0)$, $(3\pm\sqrt{3/2},1)$.
      \item Oui, on pouvait remplacer $y$ par 
      \[ y = \frac{1}{5}(x-3)^2 \]
      mais les calculs deviennent assez lourd.
      On peut aussi remplacer $x$ par 
      \[ x = 3 \pm \sqrt{5y} \] 
      mais il faut alors poser $y \geq 0$ 
      et faire attention de ne pas perdre de solutions.
    \end{enumerate}
    
  \end{solution}
    
  \item Sur le domaine de $\mathbb{R}$ déterminé par les équations
  \[f(x) = \left\{ \begin{array}{rcl}
           x_1 \geq -3/2 \\ 
           x_1 \geq x_2 \\
           x_1^2 + x_2^2 \leq 4 
            \end{array}\right.,
  \]
  trouvez \emph{tous} les minimums locaux (en justifiant) des fonctions
  \begin{enumerate}
    \item $f(x) = (x_1 - 1)^2$
    \item $f(x) = -((x_1 - 1)^2 + (x_2 - 1)^2)$
    \item $f(x) = (x_1 - 1)^2 + 2(x_2 - 1)^2.$
  \end{enumerate}
  
  \begin{solution}
    Les contraintes ainsi que le domaine admissible sont représentés
    sur le graphe ci-dessous.
    On voit immédiatement que la contrainte $c_1$ n'est jamais active,
    on sait donc que $\lambda_1$ sera nul.
    \begin{enumerate}
      \item Le minimum est atteint en $x_1 = 1$.
      On peut ensuite faire varier $x_2$ entre $-\sqrt{3}$ et $1$.
      
      \item Tout d'abord, intuitivement on remarque qu'il s'agit de
      maximiser le rayon du cercle centré en $(1,1)$.
      L'ILGCA est vérifié pour tous les points sauf $(0,0)$,
      qui n'est pas un point particulier.
      On résoud donc les conditions KKT et on obtient trois points
      \begin{itemize}
        \item $(1,1)$ qui est un maximum global,
        \item $(\sqrt{2},\sqrt{2})$ qui est un point selle,
        \item $(-\sqrt{2},-\sqrt{2})$ qui est un minimum local.
      \end{itemize}
      
      \item On voit immédiatement que $(0,0)$ est le minimum global.
      On ne pourra cependant pas le trouver avec les conditions KKT
      puisque l'ILGCA n'est pas vérifiée en ce point.
      En effet,
      \begin{align*}
        \nabla c_2 &= \begin{pmatrix} 1 \\ 0 \end{pmatrix} &
        \nabla c_3 &= \begin{pmatrix} -2x_1 \\ -2x_2 \end{pmatrix} &
      \end{align*}
      sont linéairement dépendants en $(0,0)$ (on ne tient pas compte de $c_1$ 
      puisque qu'elle n'est pas une contrainte active).
    \end{enumerate}    
    
    \begin{center}
      \begin{tikzpicture}[scale=1]
      % Axes
      \draw[axis] (-3,0) -- (3,0) node[right=\nudge cm] {\(x_1\)};
      \draw[axis] (0,-3) -- (0,3) node[above=\nudge cm] {\(x_2\)};
      \begin{scope}
        %Contraintes
        \draw[green,thick] (-3/2,-3) -- (-3/2,3) coordinate (ineq2)
          node[right=\nudge] {$c_1$};
        \draw[green,thick] (-3,-3) -- (3,3) coordinate (ineq1)
          node[below=\nudge] {$c_2$};
        \draw[green,thick] (0,0) circle (2cm) coordinate (ineq3)
          node[below=2.5cm,right=\nudge] {$c_3$};
          
        %Domaine admissible
        \begin{scope}
          \fill[hash] (0,0) -- ({sqrt(2)},{sqrt(2)}) arc (45:-135:2cm);
        \end{scope}
          
        %Solutions avec de jolies couleurs et un point tricolore
        \draw[cyan,very thick] (1,-{sqrt(3)}) -- (1,1) 
          node[left=\nudge] {(a)};
        \fill[cyan] (1,1) circle (0.1cm);
        \fill[magenta] ({sqrt(2)},{sqrt(2)}) circle (0.05cm)
          node[above=\nudge] {(b)};
        \fill[magenta] (-{sqrt(2)},-{sqrt(2)}) circle (0.05cm)
          node[left=\nudge] {(b)};
        \fill[magenta] (1,1)--(1+0.1,1) arc (0:120:0.1cm);
        \fill[yellow] (1,1)--(1+0.1,1) arc (0:-120:0.1cm)
          node[above=\nudge] {(c)};
          
        %Grid
        \draw[step=1cm,gray,very thin] (-3,-3) grid (3,3) ;
      \end{scope}
      \end{tikzpicture}
    \end{center}

  \end{solution}
  
  \item Considérez le problème d'optimisation
  \begin{align*}
    \text{min } &x_1 + x_2 \\
    \text{s.c. } &x_1^3 \geq x_2 \\
    &x_2 \geq 0.
  \end{align*}
  
  \begin{enumerate}
    \item Quelle est la solution optimale de ce problème ?
    \item Ecrivez et résolvez les conditions KKT pour ce problème. Commentez.
  \end{enumerate}
  
  \begin{solution}
    \begin{enumerate}
      \item La solution optimale est $(0,0)$.
      \item L'ILGCA est vérifiée pour les points $(x_1,x_2)$ 
      tels que $x_1 \neq 0$ puisque
      \begin{align*}
        \nabla c_1 &= \begin{pmatrix} 3x_1^2 \\ -1 \end{pmatrix} &
        \nabla c_3 &= \begin{pmatrix} 0 \\ 1 \end{pmatrix}. &
      \end{align*}
      On a donc le système suivant pour les autres points 
      \begin{align*}
        1 &= 3 \lambda_1 x_1^2 \\
        1 &= -\lambda_1 + \lambda_2 \\
        \lambda_1 (x_1^3-x_2) &\geq 0 \\
        \lambda_2 \, x_2 &\geq 0 \\
        \lambda_1 , \lambda_2 &\geq 0
      \end{align*}
      Soit $x_2 \neq 0$, alors $\lambda_2 = 0$ et $\lambda_1 = -1$,
      ce qui est impossible.
      
      Soit $x_2 = 0$, alors $\lambda_1 \, x_1^3 = 0$,
      ce qui est égalemment impossible.
      
      Notre solution optimale $(0,0)$ n'appara\^it pas ici puisqu'elle ne vérifie
      pas l'ILGCA, or c'est nécessaire pour résoudre les conditions KKT.
    \end{enumerate}
  \end{solution}
  
  \item 
  \begin{enumerate}
    \item Prouvez que si $\xopt$ est un point satisfaisant les conditions KKT
    et la condition d'indépendance des gradients, alors le vecteur des 
    multiplicateurs de Lagrage $\lambda^*$ correspondant est unique.
    \item Montrez que la reformulation d'une égalité $g(x)=0$ sous la forme
    équivalente $g(x)^2=0$ n'est pas une bonne idée du point de vue des 
    conditions d'optimalité.
  \end{enumerate}
  
  \begin{solution}
    \begin{enumerate}
      \item Si l'ILGCA est vérifié, alors, 
      par définition de l'indépendance linéaire, on a
      \begin{equation} 
        \sum_{i \in \mathcal{A}(\xopt)}{\alpha_i \nabla c_i (\xopt)} = 0
        \Rightarrow \alpha_i = 0 
        \label{eq:indepLin}
      \end{equation}
      Prouvons par l'absurde,
      soit $\lambda'$ et $\lambda''$ deux vecteurs des multiplicateurs de
      Lagrange.
      Comme les conditions KKT sont vérifiées, on a
      \[ 
        \nabla f(\xopt) = \sum_i \lambda_i' \nabla c_i = \sum_i \lambda_i'' \nabla c_i
      \]
      On peut aussi écrire 
      \[ \sum_i (\lambda_i' - \lambda_i'') \nabla c_i = 0. \]
      On montre finalement en utilisant la propriété de l'équation~\ref{eq:indepLin}
      \[ \lambda_i' = \lambda_i'' \qquad \forall i. \] 
      
      \item Le gradient de la forme équivalente est
      \[ \nabla (g(x)^2) = 2 g(x) \nabla g(x). \]
      Pour $g(x)=0$, on aura un gradient toujours nul
      et il n'y aura donc jamais d'indépendance linéaire,
      ce qui pose problème pour respecter l'ILGCA.
    \end{enumerate}
  \end{solution}
  
\end{enumerate}