\section{Conditions d'optimalité pour problèmes non contraints}
\label{sec:nonlinApe1}

Les points stationnaires correspondent aux points où 
\[\nabla f = 0.\]
Pour déterminer leurs nature, il s'agit d'analyser la hessienne de la fonction.
La hessienne est la matrice telle que
\[H_{ij}(f) = \frac{\partial^2 f}{\partial x_i \partial x_j}.\]
C'est en fait la matrice des dérivées secondes $\nabla^2 f$.

Si la hessienne est définie strictement positive
(resp. strictement négative) au point $x$, 
alors la fonction atteint un minimum (resp. maximum) local en $x$.
S'il y a des valeurs propres de chaque signe, 
alors le point stationnaire est un \emph{point selle}.

Par contre, lorsque que la hessienne a des valeurs propres nulles,
on ne sait rien dire et il faut analyser ce qu'il se passe en ce point.

Une méthode très efficace pour avoir des informations 
sur les valeurs propres (et donc le caractère) d'une matrice, 
est de s'intéresser au \emph{déterminant} et à la \emph{trace} de celle-ci.
En effet, 
\begin{align*}
  \text{tr}(M) =& \sum_i \lambda_i = \sum_i a_{ii} \\
  \text{det}(M) =& \prod_i \lambda_i.
\end{align*}

\newpage
  
\begin{enumerate}
  
  \item Soit la fonction $f(x_1, x_2)= 2 x_1^3+3 x_1^2+12x_1 x_2 + 3 x_2^2
    -6 x_2 + 54$. 
    Trouvez les points stationnaires de cette fonction et déterminez leur nature.
    La fonction possède-t-elle un minimum global? un maximum global?

    \begin{solution}
      Les points stationnaires sont $(2,-3)$ et $(1,-1)$.
      
      \[ \nabla^2 f = 6
        \begin{pmatrix}
          2x_1 + 1 & 2\\
          2 & 1
        \end{pmatrix} 
      \]
      Au point $(2,-3)$, le déterminant vaut $1$ et la trace $6$. 
      Les valeurs propres sont donc strictement positives
      et la matrice définie positive.
      C'est un minimum local.
      
      Au point $(1,-1)$, le déterminant vaut $-1$ et la trace $4$.
      Il y a une valeur propre strictement positive et une strictement négative.
      C'est un point selle.
      
      Il n'y a pas de minimum ni de maximum global puisque
      \[ \lim_{x_1 \rightarrow\pm\infty} f(x_1,x_2) = \pm\infty .\] 
    \end{solution}
    
  \item Soit la fonction
  
    \[f(x_1,x_2) = (x_1 - 1)^2 + \lambda(x_1^2 - x_2)^2\]

    pour $\lambda \in \mathbb{R}$. Trouvez les points stationnaires de $f$
    et discutez leur nature en fonction de $\lambda$.
    
    \begin{solution}
      On a ici un seul point stationnaire $(1,1)$.
      
      En $(1,1)$, on a
      \[ \nabla^2 f = 
        \begin{pmatrix}
          1+4\lambda & -2\lambda \\
          -2\lambda & \lambda
        \end{pmatrix}.
      \]
      On a donc 
      \begin{align*}
        \lambda &= \lambda_1 \lambda_2 \\
        1 + 5\lambda &= \lambda_1 + \lambda_2  
      \end{align*}
      Trois cas se présentent.
      \begin{itemize}
        \item $\lambda > 0$. La matrice est définie positive 
        donc c'est un minimum local.
        \item $\lambda = 0$. La matrice est semi-définie positive.
        La fonction devient $f(x_1,x_2) = (x_1 - 1)^2$ 
        et le point est donc un minimum global.
        \item $\lambda < 0$. La matrice est indéfinie
        donc c'est un point selle.
      \end{itemize}
    \end{solution}
    
  \item Déterminez pour chacune des fonctions ci-dessous,
  en justifiant soigneusement, les minimas locaux
  et identifiez parmi ceux-ci lesquels sont globaux 
  et lesquels sont stricts :
  \[ \text{min}_{x \in \mathbb{R}} f(x) \]
  
  \begin{enumerate}
    \item $f(x) = |x+2| + |x-3|$
      \begin{solution}
        La fonction globale devient
        \begin{align*}
          &2x-1& \forall x \geq 3 \\
          &5& \forall x \in [-2,3] \\
          &-2x+1& \forall x \leq -2.
        \end{align*}
        Les points stationnaires ($\nabla f = 0$) se trouvent en $[-2,3]$
        et la fonction prend la même valeur en chacun d'eux.
        Comme $\nabla^2 f=0$ pour tous ces points,
        il faut analyser leur comportement par rapport au reste de la fonction.
        On remarque vite (valeur absolue) 
        qu'il n'y a pas d'autres points pour lesquels
        la valeur de $f$ est inférieure.
        Ce sont donc des minimas globaux non stricts.
      \end{solution}
      
    \item $f(x) = x\sin{x} + \cos{x}$
      \begin{solution}
        Les points stationnaires vérifient
        \[x\cos{x} = 0\]
        c'est-à-dire $x=0$
        et $x= \pi/2 + k\pi$ où $(k\in\mathbb{Z})$.
        Il y a donc une alternance infinie de minimas et maximas locaux
        mais pas de globaux puisque les valeurs prises par $f$ ne cessent
        d'augmenter quand $|x| \rightarrow \infty$.
        Les minimas locaux se trouvent en $x=0$ 
        et $|x|= 3\pi/2 + 2k'\pi$ où $(k'\in\mathbb{Z}^+)$.
        Ces minimas $\xopt$ sont stricts 
        puisque on a toujours $f(\xopt) < f(x)$.
      \end{solution}
      
    \item 
    $f(x) = \left\{ \begin{array}{rcl}
             x^2-4x+2 & \mbox{lorsque} & x<1 \\ 
             x^3-5x+3 & \mbox{lorsque} & x \geq 1 \\
              \end{array}\right.$
      \begin{solution}
        On a
        \[ \nabla f(x) = \left\{ \begin{array}{rcl}
            2x-4 & \mbox{lorsque} & x<1 \\ 
            3x^2-5 & \mbox{lorsque} & x \geq 1 \\
              \end{array}\right. \]
        Note qu'il n'y a pas de discontinuité en $x=1$ puisque
        \[ \lim_{x\rightarrow 1^-} \nabla f(x)
           = \lim_{x\rightarrow 1^+} \nabla f(x) = -2. \]
        On a un minimum strict en 
        \[ x = \sqrt{\frac{5}{3}}. \]
        Les autres points pour lesquels le gradient s'annule
        ne sont pas pris en compte puisqu'il ne sont pas dans le domaine
        de leur fonction correspondante.
      \end{solution}
    
    \item $f(x) = \mbox{max}\{x^2 - 3x + 3, 3x^2 + x - 3\}$
      \begin{solution}
        La fonction peut s'écrire comme
        \[ f(x) = \left\{ \begin{array}{rcl}
            3x^2-5 & \mbox{lorsque} & \{x<-3,x>1\} \\
            x^2 - 3x + 3 & \mbox{sinon.} & \\ 
              \end{array}\right. \]
        On a donc
        \[ \nabla f(x) = \left\{ \begin{array}{rcl}
            6x+1 & \mbox{lorsque} & \{x<-3,x>1\} \\ 
            2x-3 & \mbox{lorsque} & -3<x<1 \\
            \varnothing & \mbox{sinon.} & \\            
              \end{array}\right. \]
        Cette fois-ci le gradient n'existe pas aux points de raccordement,
        il faut donc les analyser séparément.
        Tout d'abord, les points où le gradient s'annule
        ne sont pas compris dans leur domaine donc ne nous intéressent pas.
        On remarque que le gradient change de signe en $x=1$
        (négatif pour $x<1$ et positif pour $x>1$),
        on a donc un minimum strict en ce point.
      \end{solution}
              
        
          
  \end{enumerate}
  
  \item Considérez la fonction quadratique $f(x)=(1/2) x^T Q x-c^T x$ 
  avec $Q$ symétrique. Sous quelle condition cette fonction possède-t-elle un
    point stationnaire? 
    Un minimum local? 
    Un point stationnaire mais pas de minimum ni de maximum local?

    \begin{solution}
      On remarque que, comme $Q$ est symétrique,
      \[\nabla f = Qx - c.\]
      La condition pour que $f$ possède un point stationnaire
      est donc que $Qx = c$.
      
      On continue en observant que 
      \[\nabla^2 f = Q.\]
      On a donc (comme expliqué au début de la section~\ref{sec:nonlinApe1})
      un minimum (resp. maximum) local si $Q$ est définie positive (resp. négative).
      Le dernier cas est en fait un point selle
      et se produit lorsque $Q$ est indéfinie.
    \end{solution}
  
  \item Trouvez si possible une fonction $f$ de deux variables 
  et un point $\xopt$ qui maximise $f$ 
  et pour lequel $\nabla^2 f(\xopt) \succeq 0$. 
  Même question pour $\nabla^2 f(\xopt) \succ 0$.

    \begin{solution}
      Pour avoir un maximum, il faut que $\nabla f(\xopt) = 0$ 
      et $\nabla^2 f(\xopt) \preceq 0$.
      Or on demande que $\nabla^2 f(\xopt) \succeq 0$.
      Donc il faut que
      \[\nabla^2 f(\xopt) = 0.\]
      On pourrait à priori penser à une fonction de type $f(x_1,x_2) = -x_1^3-x_2^3$,
      mais on aurait alors un point selle.
      On peut par exemple choisir la fonction d'ordre 4
      \[f(x_1,x_2) = -x_1^4-x_2^4.\]
      
      Il est par contre impossible d'avoir un maximum 
      avec $\nabla^2 f(\xopt) \succ 0$.
    \end{solution}
  
  \item Démontrez les assertions suivantes en trouvant un contre-exemple
  avec $f$ une fonction de $\mathbb{R}^2 \rightarrow \mathbb{R}$
  et $\xopt = (1,1)$.
  \begin{enumerate}
    \item La condition '$\nabla f(\xopt)=0$' 
    n'est pas suffisante pour que $\xopt$ soit un minimum.
    \item La condition '$\nabla^2 f(\xopt)\succeq 0$ et $\nabla f(\xopt)=0$'
    n'est pas suffisante pour que $\xopt$ soit un minimum.
    \item La condition '$\nabla f(\xopt) = 0$ et $\nabla^2 f(\xopt)\succ 0$' 
    n'est pas nécessaire pour que $\xopt$ soit un minimum.  
    \item La condition '$\nabla f(\xopt)=0$ et $\nabla^2 f(\xopt) \succ 0$' 
    n'est pas nécessaire pour que $\xopt$ soit un minimum.
  \end{enumerate}
  
    \begin{solution}
      \begin{enumerate}
        \item $f(x_1,x_2) = (x_1 - 1)^3 - (x_2 - 1)^3$ \label{itm:power3}
        \item idem~\ref{itm:power3}
        \item $f(x_1,x_2) = (x_1 - 1)^4 + (x_2 - 1)^4$ \label{itm:power4}
        \item idem~\ref{itm:power4}
      \end{enumerate}
    \end{solution}
    
\end{enumerate}
